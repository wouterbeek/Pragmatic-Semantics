\documentclass[letterpaper]{article}

% Parts of the style depend on whether a PDF or a DVI output is created.
\usepackage{ifpdf}

\usepackage{aaai}
\usepackage{amsfonts}
\usepackage{amsmath}
\usepackage{amsthm}
\usepackage{courier}
\usepackage[english]{babel}

% Include graphics.
\usepackage{graphicx}
\ifpdf
  % Declare the supported file extensions.
  \DeclareGraphicsExtensions{.jpg,.mps,.pdf,.png}
\fi

\usepackage{helvet}
\usepackage[utf8]{inputenc}
\usepackage{times}
\usepackage{verbatim}
\usepackage{changebar}
\usepackage[normalem]{ulem}
\usepackage{url}
\newcommand{\URL}[1]{{\small \url{#1}}}

% Operator macros.
\newcommand{\absolute}[1]{\lvert#1\rvert}
\newcommand{\bigsetdef}[2]{\big\{#1\,\,\big\vert\,\,#2\big\}}
\newcommand{\card}[1]{\lvert#1\rvert}
\newcommand{\equivpair}[2]{#1 \approx #2}
\newcommand{\equivset}[1]{[#1]_{\approx}}
\newcommand{\higherapprox}[0]{\overline{\approx}}
\newcommand{\interp}[1]{#1^{\mathcal{I}}}
\newcommand{\lowerapprox}[0]{\underline{\approx}}
\newcommand{\natnum}[1]{#1 \in \mathbb{N}}
\newcommand{\pair}[2]{\langle#1,#2\rangle}
\newcommand{\powerset}[1]{\mathcal{P}(#1)}
\newcommand{\range}[2]{#1,\ldots,#2}
\newcommand{\set}[1]{\{#1\}}
\newcommand{\setdef}[2]{\{#1\,\vert\,#2\}}
\newcommand{\setrange}[2]{\{#1,\ldots,#2\}}
\newcommand{\triple}[3]{\langle#1,#2,#3\rangle}
\newcommand{\tuple}[1]{\langle#1\rangle}
\newcommand{\tuplerange}[2]{\langle#1,\ldots,#2\rangle}

% Operator declarations
\DeclareMathOperator{\indpo}{{\mathbb{IND}-\mathbb{PO}}}
\DeclareMathOperator{\indp}{{\mathbb{IND}-\mathbb{P}}}

% Theorem styles.
\newtheorem{assumption}{Assumption}
\newtheorem{axiom}{Axiom}
\newtheorem{convention}{Convention}
\newtheorem{example}{Example}
\newtheorem{definition}{Definition}
\newtheorem{lemma}{Lemma}
\newtheorem{principle}{Principle}
\newtheorem{proposition}{Proposition}
\newtheorem{specification}{Specification}
\newtheorem{statement}{Statement}
\newtheorem{theorem}{Theorem}
%\theoremstyle{definition}

\frenchspacing

\setlength{\pdfpagewidth}{8.5in}
\setlength{\pdfpageheight}{11in}

\pdfinfo{
/Title Rough Set Semantics for Identity on the Web
/Author Wouter Beek, Stefan Schlobach, Frank van Harmelen}
\setcounter{secnumdepth}{1}

\author{
  Wouter Beek \and Stefan Schlobach \and Frank van Harmelen\\
  Vrije Universiteit Amsterdam\\
  De Boelelaan 1081a\\
  1081HV Amsterdam\\
  The Netherlands
}

\title{Rough Set Semantics for Identity on the Web}

\begin{document}

\maketitle
\begin{abstract}
Identity relations are at the foundation of many
  logic-based knowledge representations.
We argue that the traditional notion of equality,
  based on Leibniz' principle of indiscernability,
  is unsuited for many realistic knowledge representation settings.
The classical  interpretation of equality is simply too strong
  when the equality statements are re-used outside their original context.

The most prominent modern example of such re-use of knowledge across contexts
  is the Linked Open Data initiative, and the Semantic Web in general.
On the Semantic Web, equality statements are used to interlink
  multiple descriptions of the same object,
  using {\small \texttt{owl:sameAs}} assertions.
And indeed, many practical uses of {\small \texttt{owl:sameAs}}
  are known to violate the formal Leibniz-style semantics.

First, we survey the problems with the classical definition of identity,
  both in general, and on the Semantic Web in particular. 
In order to provide a more flexible semantics to identity,
  we then propose a method that assigns meaning to the subrelations of
  an identity relation using the predicates that are used in a knowledge-base.
Using those indiscernability-predicates,
  we define upper and lower approximations of equality in the style of
  rought-set theory, resulting in a quality-measure for
  the identity relation.

We illustrate our approach on a realistic Semantic Web dataset,
  and we experimentally verify that our approach yields
  the intuitively desired results on this dataset.

%The full code and results of our experiments are available at
%  \URL{wouterbeek.com/identity-on-the-web} and
%  \URL{github.com/wouterbeek/IOTW}.



\end{abstract}

\section{Introduction}
\label{sec:introduction}

Identity relations are a cornerstone of logic-based knowledge representation.
They allow to state and relate properties of an object
  using multiple names for that object, and conversely,
  they allow to infer that different names actually refer to the same object.

In particular, identity relations are
  at the foundation of the Linked Open Data initiative
  and the Semantic Web (SW) in general \cite{BizerCyganiakHeath2007}.
The SW consists of many different sets of assertions
  that are published on the Web by different authors in different locations,
  often using different names for the same object.
Identity relations then allow the interlinking of these multiple descriptions
  of the same thing.
For example, statements about Amsterdam in the DBpedia dataset
  (where Amsterdam is referred to as
  \URL{http://dbpedia.org/resource/amsterdam}, abbreviated as
  \URL{dbp:amsterdam})
  can be combined with statements about Amsterdam in GeoNames
  (where Amsterdam is referred to as
  \URL{http://sws.geonames.rg/2759794}), by asserting identity between
  these two names.

However, the traditional notion of identity
  (expressed by \texttt{owl:sameAs} \cite{MotikPaterschneiderGrau2012})
  is often problematic, e.g. when objects are considered the same in some
  contexts but not in others.
The standing practice in such cases is to use weaker relations of relatedness
  (e.g., \texttt{skos:related} \cite{MilesBechhofer2009}).
Unfortunately, these relations suffer from the opposite problem of having
  almost no formal semantics, thereby limiting reasoners
  in drawing inferences.

According to the traditional semantics of the identity relation,
  identical terms can be replaced for one another in all non-modal contexts
  \emph{salva veritate}.
Practical uses of \texttt{owl:sameAs} are known to violate this
  strict condition
  \cite{HalpinHayes2010,HalpinHayesMccuskerMcguinnessThompson2010}.

\begin{comment}
The SW is not only a formal model,
  but is also a social component that evolves over time,
  i.e. it is a social machine cite{Www2013}.
Being a social and symbolic system at the same time,
  meaning on the SW is denoted by its semantics as well as its pragmatics.
\end{comment}

\subsection{Outline}

In the following section,
  we will analyze in some more detail the problems that
  are caused by the traditional notion of identity, both in general,
  and in the SW setting in particular.
After surveying existing work on these problems
  in section \ref{sec:related_work},
  we will then present our approach to the problem of identity
  in section \ref{sec:approach}.
First at a very high level and then in more formal detail.
We illustrate the results of applying our formalism to
  a realistic heterogeneous SW dataset.
In sections \ref{sec:implementation} and \ref{sec:experimental_design}
  we put our formalism to the test
  in an experimental setting,
  showing that it behaves as required.

\section{Stating the problem}
\label{sec:stating_the_problem}

Identity is often understood as the sharing of all properties between
  two objects with different names. 
This statement is known as the principle of indiscernibility
  (see principle \ref{principle:indiscernibility_of_identicals})
  and has been attributed to Leibniz \cite{Forrest2010}.\footnote{
    Inverting the implication in
      principle \ref{principle:indiscernibility_of_identicals}
      results in the identity of indiscernibles,
      which is trivially true since identity with $\phi(b)$
      is one of the properties.
    }
% According to Forrest2010 this can be found in Gottfried Wilhelm Leibniz,
% Discourse on Metaphysics, section 9.
% There I find the following passage:
% \begin{quote}
% That every individual substance expresses the whole universe
%   in its own manner and that in its full concept is included
%   all its experiences together with all the attendant circumstances
%   and the whole sequence of exterior events.
% \end{quote}

\begin{principle}[Indiscernibility of identicals]
\label{principle:indiscernibility_of_identicals}
\begin{equation}
    a = b
  \rightarrow
    \forall \phi \in \Phi (\phi(a) = \phi(b)\nonumber)
\end{equation}
\end{principle}

\noindent Although the principle provides necessary and sufficient conditions
  for identity, it does not point toward an automated procedure
  for enumerating the extension of the identity relation.
Due to its circular nature, the set of properties includes
  ``being identical to $x$'' (for every object $x$).
Even though this principle does not
  allow a positive identification of identity pairs,
  it does provide an exclusion condition;
  namely objects that are known to not share some property
  are also known to not be identical.

\subsection{Generic problems of identity}

Identity poses several problems that are not specific to the SW.
Firstly, identity does not hold across (all) modal contexts,
  allowing Lois Lane to believe that Superman saved her,
  without knowing that Clark Kent saved her.

Secondly, identity seems to be a context-specific concept \cite{Geach1967},
  allowing two medicines to be the same chemical substance
  while not being the same commercial drug.

Thirdly, identity over time poses problems,
  since a ship may be considered the same
  even though all the components from which it is built
  have been changed over the course of time \cite{Lewis1986}.

Lastly, there is the problem of identity under counterfactual assertions
  such as ``If Obama would have been born outside the US,
  then he would not have been president of the US today.''\cite{Kripke1980}

\subsection{SW-specific problems of identity: Semantics}

Besides the generic problems of identity,
  there are problems specific to the SW.
The first SW-specific problem follows from its semantics;
  the second follows from its pragmatics.
The semantics for SW identity are given in definition \ref{def:owl_sameAs}.

\begin{definition}[Semantics of {\small \texttt{owl:sameAs}}]
\label{def:owl_sameAs}
\begin{equation}
    \langle a, b \rangle \in Ext(I({\small \texttt{owl:sameAs}}))
  \,\iff\,
    a = b\nonumber
\end{equation}
\end{definition}

\noindent In the context of the SW,
  identity assertions are extra strong because of the Open World Assumption.
Stating that two objects are the same
  implies that from now on no new property can be stated about
  only one of those objects
  (this follows from definition \ref{def:owl_sameAs} in combination with
  the principle of substitutivity \emph{salva veritate}).
  % Look this one up in
  % W.V.O. Quine, Quintessence, extensions, Reference and Modality, p. 378.
  % Looked this one up, interesting, but is not directly applicable here.
For instance, when one SW contributor claims that
  medicines $a$ and $b$ are the same
  based on them having the same chemical composition,
  this prohibits another SW contributor from stating that
  $a$ and $b$ are produced by different companies
  without her introducing an inconsistency.
Formulated in terms of
  principle \ref{principle:indiscernibility_of_identicals},
  on the SW the set $\Phi$ contains
  all properties that can possibly be expressed
  in the modeling language.
This set is obviously much bigger than the actual vocabulary
  of any in-use dataset.

Moreover, whether or not two objects share the absence of a property
  (i.e., a property of the form ``does not have the property $\phi$''),
  cannot be concluded based on the absence of a property assertion.
Such `negative knowledge' must be provided explicitly
  using class restrictions.

\subsection{SW-specific problems of identity: Pragmatics}

When we take the social component of the SW into account,
  we observe that modelers have different opinions about
  whether two objects are identical or not,
  because they operate in different contexts.
This is unlike many traditional uses of knowledge-bases,
  where a knowledge-base is rarely re-used outside its original
  context of construction.
As a consequence, SW modelers are known not to conform to
  the strict semantics of identity (definition \ref{def:owl_sameAs}),
  resulting in situations where one person
  claims two objects are identical
  whereas another considers them to be only (closely) related.
It is indeed unrealistic that all who contribute to the SW
  are able to quantify over the entire realm of possible properties
  (i.e., $\Phi$ in principle \ref{principle:indiscernibility_of_identicals}).

The SW community recognizes this tension between
  the semantics and pragmatics of identity.
At the Semantics for Big Data track at the AAAI Fall Symposium 2013
  \cite{SemanticsBigData2013}
  the problem of identity was considered one of the most
  pressing problems facing the Semantic Web today.

The pragmatics of the SW is encoded in the open-ended collection of
  common practices that effectuate the creation and usage of SW content.
An example of such an encoded common practice is the five `stars'
  of Linked Open Data (LOD) publishing \cite{Bernerslee2010}.
These five `stars' are effectively five \emph{maxims} that specify
  (part of) the pragmatics of LOD publishing,
  much like the Gricean maxims specify
  (part of) the pragmatics of natural language discourse \cite{Grice1989}.
The fifth `star' or maxim is given in \ref{eq:data_linking_maxim}.
\begin{principle}[Data linking maxim]
  \label{eq:data_linking_maxim}
  \begin{quote}
    Link your data to other people's data to provide context.
  \end{quote}
\end{principle}

\noindent Given that RDF links are often specified by using
  the {\small \texttt{owl:saveAs}} predicate term \cite{Void2011}
  % The VoID W3C Interest Group Note states that
  % ``RDF links often have the \texttt{owl:sameAs} predicate.''
  we conclude that the pragmatics of the SW
  aggravates the already problematic semantics of identity.

The pragmatics of the SW, observed in contemporary common practices,
  states that by asserting identity between objects,
  more context is added for those objects,
  since such identity links result in more facts being asserted
  about the same resource.
At the same time we see that the strict semantics of identity
  gets violated once the context in which an object was created
  is extended -- by identity assertions -- beyond its original context
  (see the section on the generic problems of identity).
This is true regardless of whether `context' is defined in terms of
  `intended use', `domain', `time', or `modality'.

Concluding, from the social point of view the requirements on SW modelers
  are unreasonably high when they are required to
  assert identity links in accordance with the semantics.
At the same time the pragmatics states that modelers should
  make those links in order to place their knowledge into context.



\subsection{Research goals}
\label{sec:research_goals}

Based on the above analysis, we can state the following desiderata for 
an identity relation that does not suffer from the problems stated above: 

\begin{enumerate}
\item In an identity relation the pairs all look the same.
      We want to characterize subrelations of an identity relation in terms
      of the predicates that are important in a particular context.
\item Based on an existing identity relation we want to give semantically
      motivated suggestions for extending or limiting the identity relation.
\item We want to assess the quality of an identity relation based on
      the consistency with which it is applied to the data.
\end{enumerate}

\section{Related work}
\label{sec:related_work}

Existing research suggests six different solutions for
  the problem of identity on the SW.

\textbf{[1] Introduce weaker versions of {\small \texttt{owl:sameAs}}}
  \cite{HalpinHayes2010,MccuskerMcguinness2010}.
Candidates for replacement are
  the SKOS concepts
  {\small \texttt{skos:related}} and {\small \texttt{skos:exactMatch}}
  \cite{MilesBechhofer2009}.
The former is not transitive,
  thereby limiting the possibilities for reasoning.
The latter is transitive,
  but can only be used in certain contexts.
It is not defined in what contexts it can be used
  \cite{MilesBechhofer2009}.\footnote{
    For instance, the property {\small \texttt{skos:exactMatch}}
    ``is used to link two concepts, indicating a high degree of confidence
    that the concepts can be used interchangeably across a wide range of
    information retrieval applications.''
  }
\begin{comment}
% SIMILARITY
The problem with using weaker notions such as relatedness,
  is that everything is related to everything in \emph{some} way.}
% Shall we discuss similarity here as well?
% Does similarity differ from relatedness?
\end{comment}

\textbf{[2] Restrict the applicability of identity relations}
  to specific contexts.
In terms of Semantic Web technology, identities are expected to hold
  within a named graph or within a namespace,
  but not necessarily outside of it \cite{HalpinHayes2010}.
\cite{Melo2013} has successfully used the Unique Names Assumption
  within namespaces in order to identify many (arguably) spurious
  identity statements.

\textbf{[3] Introduce additional vocabulary} that does not weaken but extends
  the existing identity relation.
\cite{HalpinHayes2010} mention an explicit distinction that could be made
  between mentioning a term and using a term,
  thereby distinguishing an object and a Web document describing that object.
Other possible extensions of {\small \texttt{owl:sameAs}} might take
  the Fuzzyness and/or uncertainty of identity statements into account.

\textbf{[4] Use domain-specific identity relations}
  \cite{MccuskerMcguinness2010}.
For instance
    ``$x$ and $y$ have the same medical use''
  replaces
    identity in the domain of medicine,
and
    ``$x$ and $y$ are the same molecule''
  replaces
    identity in the domain of chemistry.
The downside to this solution is that domain-specific links are
  only locally valid, thereby limiting knowledge reuse.

\textbf{[5] Change the modeling practice}, possibly in a (semi-)automated way
  by adapting visualization and modeling toolkits to produce notifications
  upon reading SW data, or by posing additional restrictions on the creation
  and alteration of data. For example, adding an RDF link could require
  reciprocal confirmation from the maintainers of the respective datasets.
  \cite{HalpinHayes2010,DingShinavierFininMcguinness2010}
The problem with introducing checks on editing operations,
  is that it violates one of the fundamental underpinnings of the SW;
  namely that on the Web of Data anybody is allowed to say
  anything about anything \cite{AntoniouGrothHarmelenHoekstra2012}.

\textbf{[6] Extract network properties of {\small \texttt{owl:sameAs}}
  datasets} \cite{DingShinavierShangguanMcguinness2010}.
Although this work shows that network analysis can provide insights
  into the ways in which identity is used in the SW,
  these endeavors have not yet been related to the semantics of the
  identity relation.
We believe that utilizing network theoretic aspects in order to
  determine the meaning of identity statements
  would be interesting future research.

What the existing approaches have in common is
  that quite some work has to be done
  (adapting or creating standards, instructing modelers, converting existing
  datasets) in order to resolve only some of the problems of identity.
Our approach provides a way of dealing with the heterogeneous real-world
  usage of identity in the SW that is fully automated and requires
  no changes to standards, modeling practices, or existing datasets.


\section{Approach}
\label{sec:approach}

First we give a short outline of our approach and then
  we provide a more detailed description of the individual steps.

\subsection{Outline of the approach}

We start by assuming that we are given an identity relation $\approx$.
We will then reinterpret this relation as an indiscernibility relation
  relative to different sets of predicates.
Pairs that have the same indiscernibility predicates
  are simi-discernible, i.e.: they discern resources
  based on the same criteria.
Simi-discernibility is an equivalence relation
  which partitions all pairs and thus also the identity relation $\approx$.
The members of the indiscernibility partition
  have a certain overlap with the original identity relation.
The overlap between an indiscernibility subset and the identity relation
  is called an \emph{identity subrelation}.
Each identity subrelation is characterized in terms of predicates
  from the domain vocabulary.
Different forms of identity can therefore be distinguished
  and meaningfully described.
Based on whether there is a complete or a partial overlap
  between the simi-discernible partition members and
  the identity subrelations,
  these partition members belong either to the lower ($\lowerapprox$)
  or to the higher approximation ($\higherapprox$) of $\approx$.
Besides setting a lower and a higher bound to the identity relation,
  we can also calculate the quality of the identity relation
  and the precision of each identity subrelation.

\subsection{Preliminaries}
\label{sec:preliminaries}

%TODO 'resource'->'object'
%TODO explain that 'objects' are not 'object terms'
%     and are not even (always) denoted by object terms.

$G$ denotes an RDF graph. It consists of a set of ground binary
predicates $p(s,o)$, called ``triples'' in Semantic Web jargon, and often
written as $\triple{s}{p}{o}$. These triples form a graph with all
subjects $s$ and objects $o$ as nodes, and 
each assertion $p(s,o)$ corresponding to a directed edge labelled $p$
between $s$ and $o$. 
We assume that graphs are always closed under
  RDFS and OWL-DL entailment.

We identify subsets of RDF terms based on
  their positional occurrence in triples in $G$:
  $S_G$, $P_G$, and $O_G$ denote the subject, predicate and object terms
  in $G$ respectively.

The interpretation $I$ maps RDF terms onto resources,
  and triples onto truth values.
The extension function $Ext$ maps resources onto pairs of resources.
$I(\triple{s}{p}{o})$ is true iff
  $\pair{I(s)}{I(o)} \in Ext(I(p))$ \cite{Hayes2004}.

A note on terminology: Our use of the word `property' does not coincide
  with the notion of an RDF property, but is closer to the notion
  of a FOL property. Our use of the word `relation' will be close to
  both the notion of a FOL relation and the notion of an RDF property.
  We briefly explain the distinctions.

An RDF property is a resource (i.e., a member of the domain)
  that is the interpretation of
  an RDF term that occurs in the predicate position of a triple.\footnote{
    Notice that we do not use the possibility modality in this formulation.
    Every URI \emph{could} occur in the predicate position of a triple,
      but some URIs are known to denote non-property resources.
  }
As the example above shows, the extension of an RDF property is
  a binary \emph{FOL relation},
  i.e., a subset of the Cartesian product of the domain.\footnote{
    In RDF the domain is the set of RDF resources.}

A \emph{FOL property} is a subset of the domain.
The correlate of a FOL property in RDF is a pair that consists of
  a predicate term and an object term (in that order).
The extension of the interpretation of such a pair $\pair{p}{o}$
  is $Ext(I(p))(I(o))$, which is -- indeed -- a set of RDF resources.

In section \ref{sec:path_expressions} we will generalize
  both the concept of a property and the concept of a relation.
Generalized properties are similar to FOL properties
  and generalized binary relations are similar to binary FOL relations,
  but the latter do not correspond to a (single) RDF property.

\begin{comment}
$\equivset{x}$ is the equivalence class for $x$
  under equivalence relation $\approx$.
\end{comment}


\subsection{Reinterpreting identity as indiscernibility}

We start by assuming that we are given an identity relation $\approx$,
  which partitions the subject terms $S_G$ according to
  \mbox{equation} \ref{eq:equivalence_set}.

\begin{equation}
\label{eq:equivalence_set}
  \equivset{x}
=
  \setdef{
    y \in S_G
  }{
    \equivpair{x}{y}
  }
\end{equation}

\noindent Identity can be defined as the smallest equivalence relation,
  i.e. the most fine-grained partition of $S_G$.
For reasoning purposes, the fact that $\approx$ is an equivalence relation
  is important, since this allows for both symmetrical
  and transitive inference.
As we saw in principle \ref{principle:indiscernibility_of_identicals},
  identity implies indiscernibility with respect to all properties.

We can generalize the notion of indiscernibility
  by parameterizing the set of properties with respect to which
  indiscernibility is determined.
According to this generalization,
  resources $x$ and $y$ are indiscernible with respect to
  a set of properties $PO \subseteq P_G \times O_G$
  iff $\forall po \in PO (po(x) \leftrightarrow po(y))$ is the case.

It is important to note that every indiscernibility relation
  is also an equivalence relation, although not necessarily the smallest one.
Moreover, every indiscernibility relation defined over the domain $S_G$
  is also an identity relation,
  just over a different domain \cite{Quine1950}.\footnote{
    For instance, the set of properties
      ``has an income of $x$ euro's a month''
      with $x$ a decimal number, does not uniquely identify people
      (since two people may have the same income),
      but does uniquely identify income groups.
    }

We now reinterpret the identity relation $\approx$,
  as if it were an indiscernibility relation
  whose set of properties $PO$ is implicit.
Based on the extensional specification of the identity relation,
  we can make the set of properties with respect to which
  it is indiscernible explicit.\footnote{
    In Formal Concept Analysis (FCA) literature this set of properties
    is called the intension. The relation between extension and intension
    in FCA is similar to the relation between subject terms and properties,
    but the similarity becomes less apparent when we extend our approach
    to pairs of subject terms (as the `extension') and sets of
    shared predicate terms (as the `intension') later in this section.
    Our thanks go to Cliff Joslyn for pointing this out.
  }
Definition \ref{def:indiscernibility_properties} makes explicit
  the properties relative to which the terms $x_i$ are indiscernibile
  .

%TODO Define for sets or for pairs?
\begin{definition}[Indiscernibility properties]
\label{def:indiscernibility_properties}
\begin{align}
  \indpo_{\approx}(\set{\range{x_1}{x_n}})
=
  \setdef{
    \pair{p}{o} \in P_G \times O_G
  }{\nonumber\\
    \bigwedge_{1 \leq i \leq n}
      \exists p_i \in \equivset{p},
        \exists o_i \in \equivset{o}(
          \triple{x_i}{p_i}{o_i} \in G
        )
% Alternatively, we can place the quantifies outside the conjunction:
%    \exists_{\range{p_1}{p_n} \in \equivset{p}},
%      \exists_{\range{o_1}{o_n} \in \equivset{o}}
%        \bigwedge_{1 \leq i \leq n} \triple{x_i}{p_i}{o_i} \in G
  }\nonumber
\end{align}
\end{definition}

\noindent Using definition \ref{def:indiscernibility_properties},
  we can deduce that the indiscernibility properties for
  the identity pair
  {\small $\pair{\texttt{dbp:Amsterdam}}{\texttt{dbp:Berlin}}$}
  would contain the properties
  ``is a city'' and ``is located in Europe''.

Notice that in definition \ref{def:indiscernibility_properties}
  we close both the predicate terms $p$ and the object terms $o$
  under identity.\footnote{
    It is a common modeling practice in OWL to use equivalence
      instead of identity statements between RDF properties
      (i.e. {\small \texttt{owl:equivalentProperty}})
      and OWL classes
      (i.e. {\small \texttt{owl:equivalentClass}}).
  }
Performing these closures is important in order to identify
  the relevant indiscernibility properties.
Suppose that the RDF resources {\small \texttt{dbp:Berlin}}
  and {\small \texttt{fb:m.0156q}} are identical,
  as are the RDF properties {\small \texttt{schema:containedIn}} and
  {\small \texttt{dbp-owl:locatedInArea}}.
If the database also contains the triples in
  example \ref{ex:triple1} and \ref{ex:triple2},
  we want to be able to identify the pairs in
  {\small $\set{\texttt{schema:containedIn},\texttt{dbp-owl:locatedInArea}}$}
  {\small $\times \set{\texttt{dbp:Berlin},\texttt{fb.0156q}}$}
  as indiscernibility properties.

\small
\begin{example}[Triples]
\begin{align}
\langle
  \texttt{fb:Teufelsberg},\label{ex:triple1}\\
  \texttt{schema:containedIn},\,
  \texttt{dbp:Berlin}
\rangle\nonumber\\
\langle
  \texttt{dbp:Teufelsberg},\label{ex:triple2}\\
  \texttt{dbp-owl:locatedInArea},\,
  \texttt{fb.0156q}
\rangle\nonumber
\end{align}
\end{example}
\normalsize

\noindent In the above, we were interested in the properties
  that resources share with one other
  (e.g., being a city, being located in Europe).
But we are also interested in the predicates that are shared by
  a set of resources.
This amounts to a simple abstraction of
  definition \ref{def:indiscernibility_properties},
  equating the sets of objects (closed under identity)
  and only returning the set of shared RDF predicate terms
  (see definition \ref{def:indiscernibility_predicates}).

\small
\begin{definition}[Indiscernibility predicates]
\label{def:indiscernibility_predicates}
\begin{align}
  \indp_{\approx}(\setrange{x_1}{x_n})
=
  \setdef{
    p \in P_G
  }{
    \exists_{\range{p_1}{p_n} \in \equivset{p}}(\nonumber\\
        \equivset{
          \setdef{
            o \in O_G
          }{
            \triple{x_1}{p_1}{o}
          }
        }
      =
        \ldots
      =
        \equivset{
          \setdef{
            o \in O_G
          }{
            \triple{x_n}{p_n}{o}
          }
        }
    )
  }\nonumber
\end{align}
\end{definition}
\normalsize

\noindent Taking the same example as before, the identity pair
  {\small $\pair{\texttt{dbp:Amsterdam}}{\texttt{dbp:Berlin}}$}
  has the identity closure of ``is a'' and ``is located in''
  as its indiscernibility predicates.



\subsection{Discerning the same}

In the previous section we saw that resources are \mbox{\emph{indiscernible}}
  with respect to $PO$ iff they cannot be told apart
  in a language that only contains the properties denoted by $PO$
  (the so-called indiscernibility properties):

In the same vein,
  and builing upon definition \ref{def:indiscernibility_predicates},
  we say that two pairs of resources are \emph{simi-discernible}
  iff their \mbox{indiscernibility} predicates $P \subseteq P_G$ are the same.

For example, the pair {\small $\pair{\texttt{Amsterdam}}{\texttt{Berlin}}$}
  is simi-discernible to the pair
  {\small $\pair{\texttt{Nanjing}}{\texttt{Dalian}}$},
  because both pairs have the same set of discernibility predicates,
  namely ``is a'' and ``is located in''.

When we look at the pairs that constitute (the extension of)
  an identity relation, all identity assertions look the same.
But when we take the considerations of the previous section into account,
  we see that within a given identity relation
  there are pairs that assert indiscernibility
  based on different domain predicates.
Stating this formally,
  simi-discernibility is an equivalence relation on pairs of resources,
  which induces a partition of the Cartesian product of the domain.
Definition \ref{def:simidiscernibility_relation} makes this concrete
  in terms of the earlier definitions.

\begin{definition}[Simi-discernibility relation]
\label{def:simidiscernibility_relation}
\begin{align}
  \equiv_{\indp_{\approx}}
=
  \setdef{
    \pair{\pair{x_1}{x_2}}{\pair{y_1}{y_2}} \in (S_G^2)^2
  }{\nonumber\\
    \indp_{\approx}(\set{x_1,x_2}) = \indp_{\approx}(\set{y_1,y_2})
  }\nonumber
\end{align}
\end{definition}



\subsection{Partitioning identity}

The members of the partition induced by $\equiv_{\indp_{\approx}}$
  are sets of resource pairs that share the same sharing properties.\footnote{
    We briefly reflect on our wording here.
    \emph{Within} a resource pair, what is shared are properties
      (denoted by a pair of an RDF predicate term and an RDF object term,
      in that order).
    \emph{Between} pairs, what is shared -- if something is shared --
      are predicates, denoted by RDF predicate terms.
  }

Notice that the partitioned pairs contain but are not limited to
  the identity pairs.
Therefore, for sets of pairs closed under simi-discernibility
  we have the following three possibilities:
  \begin{enumerate}
    \item All pairs are identity pairs.
          This characterizes a consistent portion of the identity relation,
          since no simi-discernible pair is left out of this set.
    \item Some pairs are identity pairs.
          This characterizes a portion of the identity relation which is not
          applied consistently with respect to
          the simi-discernibility relation.
    \item No pairs are identity pairs.
          This characterizes a portion of the collection of pairs
          that is consistently kept out of the identity relation.
  \end{enumerate}

\noindent Each member of the simi-discernibility partition that is not
  of the third kind, i.e. every set of pairs that contains some identity pair,
  can be though of as an identity subrelation.
The simi-discernibility partition also partitions the identity relation
  into \emph{identity subrelations}.
Each identity subrelation can be described in terms of
  its discernibility predicates,
  i.e. in meaningful terms drawn from the domain vocabulary.

For instance, in the IIMB dataset that we will use in our experiment
  in sections \ref{sec:implementation} and \ref{sec:experimental_design}
  there are some identical resources that share
  the property {\small \texttt{IIMBTBOX:spoken\_in}},
  while other pairs share the property
  {\small \texttt{IIMBTBOX:form\_of\_government}}.
As a matter of fact, the set of pairs of resources that are spoken in
  the same language (i.e. movies) are disjoint from
  the set of pairs of resources that have the same form of government
  (i.e. countries), indicating identity according to
  different indiscernibility criteria.
In this example, one subset of the identity relation does not discern
  resources that are spoken in the same language,
  whereas another does not discern resources that have
  the same form of government.



\begin{comment}
Fig. 1 shows an example
  of a discernibility partitioning for a given identity relation.

We can thus identify subsets of an identity relation based on
  differences in the sets of properties relative to which
  the resource pairs that they consist of are
  (in)discernible from one another.
Identity of indiscernibility criteria provides
  another equivalence relation
  ($\approx_{\indp}$, def. \ref{def:indiscernibility_partition}),
  that partitions the identity relation $\approx$ into
  identity subrelations that characterize identity based on different
  indiscernibility criteria.

\begin{figure*}
\label{fig:iimb_example}
\centering
\includegraphics[width=\textwidth]{iimb_approximation_example_crop}
\caption{
  An example of a discernibility partition for an identity relation
    consisting of 365 pairs applied to the fourth IIMB linkset.
  Each node is annotated with the set of predicates $P$ for which
    its pairs are $P$-indiscernible.
  The number of identity pairs within each partition set
    is displayed to the right of the predicate set label.
  Partition sets that contain no identity pair are not show.
  The number that occurs to the left of the predicate label in each node
    indicates how may pairs in that node are identity pairs.
  The lower approximation consists of the nodes with a solid border,
    indicating that they contain only identity pairs.
  The higher approximation consists of all displayed nodes.}
\end{figure*}
\end{comment}


\subsection{Quality \& Approximation}
\label{sec:approximation}

Not all identity subrelations have the same quality.
Indeed, when we look at the subdivision into three `categories' above,
  we are able to distinguish between a lower approximation of identity,
  as the union of subrelations from the first category
  (definition \ref{def:identity_lower_approximation}),
  and a higher approximation of identity,
  as the union of subrelations from both the first and the second category
  (definition \ref{def:identity_higher_approximation}).

\begin{definition}[Lower approximation]
\label{def:identity_lower_approximation}
\begin{align}
  x \in \lowerapprox
\, \iff \,
    \setdef{y}{x \equiv_{\indp_{\approx}} y}
  \; \subseteq \;
    \approx\nonumber
\end{align}
\end{definition}

\begin{definition}[Higher approximation]
\label{def:identity_higher_approximation}
\begin{align}
  x \in \higherapprox
\, \iff \,
      \setdef{y}{x \equiv_{\indp_{\approx}} y}
    \, \cap \,
      \approx
  \; \neq \;
    \emptyset\nonumber
\end{align}
\end{definition}

\noindent Based on these approximations we can give
  the rough set representation $\pair{\lowerapprox}{\higherapprox}$
  of identity relation $\approx$.
The quality of a rough set representation is given in
  definition \ref{def:quality}.
The intuition behind this quality measure is that the crispness
  of a set should be proportional to the quality
  of the identity relation on which it is based.
Since a consistently applied identity relation has relatively many
  partition sets that contain either
  no identity pairs (small value for $\lowerapprox$) or
  only identity pairs (large value for $\higherapprox$),
  a more consistent identity relation has a higher accuracy.

\begin{definition}[Quality]
\label{def:quality}
\begin{align}
  \alpha(\approx)
\, = \,
  \dfrac{
    \card{\underline{\sim}}
  }{
    \card{\overline{\sim}}
  }\nonumber
  %\card{\lowerapprox} / \card{\higherapprox}
\end{align}
\end{definition}





\section{Implementation}
\label{sec:implementation}

Using our operationalization of the five star model of data sharing
 in section \ref{sec:operationalization} we created
 a web observatory for linked open data,
 called \obs.\footnote{Code and results available at
   \url{https://github.com/wouterbeek/LODObs}}

\obs uses an automated script in order to look for
 data on the Web.
The script must be given a set of locations where Web data
 is likely to be found.
For this we use CKAN API.\footnote{\url{http://docs.ckan.org}}
CKAN is an open-source data portal platform
 that allows datasets that are published on the Web to be catalogued.
There are various catalogues available,
 including the governmental initiatives towards data sharing
 of the UK and the USA.

\obs gives a tabular overview of the results of
 processing the various resources, see Figure \ref{fig:lod_observer}.
\obs provides more detailed information than we are able to give
 in our results section (i.e. section \ref{sec:results});
 e.g. it shows the specific error messages encountered
 and the syntax error that occur while reading a serialization format.

\begin{figure*}[th!]
  \label{fig:lod_observer}
  \centering
  \includegraphics[width=0.85\textwidth]{./img/table}
  \caption{
    A part of the table that is generated by \obs.
    Each row in the table displays the results of retrieving a single
     resource from Datahub.
    Each column in the table corresponds to a specific action
     that is performed for a specific resource.
    The picture shows the results of executing the following four actions
     for three resources:
     (1) downloading the resource,
     (2) unarchiving it,
     (3) determining the file size, and
     (4) counting some basic VoID statistics \cite{Void2011}.
  }
\end{figure*}

\subsubsection*{Locating resources}

CKAN uses URLs as universal locators for resources.
URLs are URIs that identify a resource via a representation of
 its primary access mechanism or scheme (e.g. HTTP)
 and its network location (e.g. \texttt{www.datahub.io})
 that can be accessed using standardized operations
 defined for that scheme. \cite{Rfc3305}

The datasets that are catalogued by CKAN provide
 a starting point for an automated agent to search for data on the Web,
 since all data registrations can be queried using an API.
Since datasets have to be explicitly registered at a CKAN catalogue,
 this ensures that they are purposefully published
 as machine-processable data,
 thereby following our definition of a `resource'
 (see section \ref{sec:operationalization})
Each CKAN resource has a URL property field,
 which allows an automated agent to find a URL string for each resource.

\subsubsection*{Connecting to disseminating host}

When using the URL string to locate a resource we find that
 not all URL strings parse according to the grammar
 in the URL specification \cite{Rfc3986},
 and some URLs that are grammatically correct contain a scheme
 that is not registered with
 the Internet Assigned Numbers Authority
 (IANA).\footnote{\url{http://www.iana.org/assignments/uri-schemes/uri-schemes.xhtml}}
%\footnote{
%  The number of resources for which the URL string parses correctly
%  can be increased by trimming leading and trailing whitespace (US-ASCII 32)
%  and by prepending the string with a commonly occurring scheme descriptor
%  (e.g., \texttt{http://}) in case the grammar cannot detect a scheme.
%}

Culprits:
\begin{itemize}[noitemsep,nolistsep]
  \item The URL string does not parse according to the RFC grammar.
  \item The parsed URL string does not contain an IANA-registered scheme.
\end{itemize}

For URLs that are grammatically correct an automated agent is able to
 verify whether there exists a host authority at the location
 denoted by the URL.
As with the Web of Documents, this is not always the case,
 resulting in a ``host not found exception''.
Once a host authority is found, it has to accept a connection
 with the automated agent.
Only when a connection is established can the agent send its request
 to the authority.
The agent and authority both have to maintain the connection
 for the duration of the subsequent request/response-interaction.

Culprits:
\begin{itemize}[noitemsep,nolistsep]
\item The host that is denoted by the URL's authority string cannot be found.
\item The connection was refused by the host.
\item The connection was neither refused nor accepted by the host.
\item The connection was established, but was broken off
      during subsequent communication.
\item The host was redirecting the connecting agent indefinitely.
\end{itemize}

\subsubsection*{Retrieving resource from host}

Once a reliable connection between agent and host is established
 for the duration of a communicative interaction,
 the agent is able to send a request in one of
 the standardized Internet communication protocols.
Specifically, \obs supports communications via FTP and HTTP(S).

Various things can go wrong in both formulating the request and
 in replying to it.
This results in various status codes that denote different problems
 that cause the communication to be ineffective.

\subsubsection*{Open license}

Many CKAN registered resources have a license property.
The licenses are similar to those defined by the Open Data Commons,
 but some mismatches occur.
In the Datahub CKAN repository, 33 licenses are used,
 18 of which are underdefined (i.e., with no semantic description),
 impacting 5\% of the resources.
We have added manual mappings from the CKAN licenses onto
 the repository of Open Data Commons license descriptions.
%This results in added descriptions for 14 underdefined licenses,
% additional properties for 4 licenses that were already partially defined,
% and leaves only 2 underdefined licenses that have been equated to
% `license' \texttt{ckan:None} using \texttt{owl:sameAs} statements.

Culprits:
\begin{itemize}[noitemsep,nolistsep]
\item Has no license string.
\item Has a license string that cannot be mapped to
       a license that occurs in
       the Open Data Commons license description set.
\item Has a recognized license that is not open.
\end{itemize}

\subsubsection*{Structured \& non-propetary}
\label{sec:implementation_mime}

In order to implement the structuredness requirement,
 we have manually classified the MIME types that occur in the CKAN catalogue
 into `structured' and `unstructured' ones.
This goes against our interpretation of structuredness as
 a gradient property, but the partial order constituted by the relation
 ``supports at least the same set of relational operators''
 cannot be easily established in an automated way,
 as this would require a model of query operators
 and a description of MIME types in terms of the operators that are
 supported by those content types.

We list the 4 ways in which we can check for a resource's content type
 in CKAN, annotated with the number of resources for which
 this type can be retrieved:
\begin{itemize}[noitemsep,nolistsep]
  \item The value of the CKAN \texttt{mimetype} property.
        Present for 6,332 resources (45\%).
  \item The value of the CKAN \texttt{format} property
        Present for 10,226 resources (73\%).
  \item The MIME type in the HTTP \texttt{Content-Type} response header
         (not present in CKAN).
  \item The extension of the resource file.
\end{itemize}

We are specifically interested in linked data,
 but not all linked data serialization formats have a MIME type
 that is registered by IANA.
Moreover, some of the registered linked data content types are deprecated.
We take both deprecated and current MIME types into account,
 and officially registered ones as well as ones that are
 de facto being used to denoted linked data.
Not all MIME types that occur in CKAN repositories are valid,
 some of them seem to be typos/variants of existing MIME types
 for which we have added mappings manually.

The values of the CKAN \texttt{format} property are not standardized
 and are also manually mapped onto IANA-registered and de facto used
 MIME types.
Some of the format values seem to be typographic variants
 of each other (impacting 96 resources).
For some format values no mapping to an existing MIME type
 could be found (impacting 52 resources).

%The MIME type that occurs in the \texttt{Content-Type} response header
% is not always the same as the MIME type denoted by either the
% \texttt{format} or the \texttt{mimetype} property.
%Sometimes this is more generic or a more specific than
% the CKAN-registered value (XX\%),
% but sometimes it is another value altogether.

File extensions were not mapped to content type,
 because of the absence of a straightforward mapping.
We do not believe file extensions are a reliable indicator
 of a resource's format.

A special case occurs for resources that are compressed
 in some archive format.
For these neither MIME type, format, nor Content-Type header
 are indicative of the uncompressed content,
 so for these we have to rely on the file extension.

When none of the above enumerated methods works,
 we can try to parse a file's first few lines.
This is generally quite difficult because of the large number
 of different formats, encoding types, and syntactic error that may occur.
In the general case we are only able to make a best guess at
 a resource's format.

Culprits:
\begin{itemize}[noitemsep,nolistsep]
  \item The resource's type is not set.
  \item The resource's type is set, but it does not map to a MIME type
         that is registered by IANA and it is not one of the MIME types
         in the list of de facto used identifiers of LOD content.
  \item The resource's type can be mapped to an IANA-registered type
         or a de facto LOD type, but it does not denote structured data.
  \item The resource's type can be mapped to an IANA-registered type
         or a de facto LOD type, but it denotes a proprietary format.
\end{itemize}

\subsubsection*{Syntactic correctness: readable triples}

We use SWI-Prolog's Semweb library \cite{wielemaker2003}
 for loading the resources into a triple store.

The number of triples that can be loaded is often inconsistent with
 the value of the CKAN \texttt{size} property.\footnote{The CKAN
    \texttt{size} property is often taken to represent the actual number
    of triples in a dataset.
   For example, the famous visualizations of the LOD cloud make use of
    the values for this property.
   Our observatory shows that these visualizations are not always based
    on the correct numbers.}

Culprits:
\begin{itemize}[noitemsep,nolistsep]
  \item From the resource no triples can be read.
  \item From the resource some triples are read (syntax errors).
  \item From the resource all triples are read (no syntax error).
\end{itemize}


\section{Experimental Validation}
\label{sec:experiment}
\label{sec:experimental_design}
\label{sec:experimental_validation}

In order to subject our theory to an experimental test,
  we use the IIMB dataset\footnote{\URL{islab.di.unimi.it/iimb}}
  used in the instance matching track of the
  2012 Ontology Alignment Evaluation Initiative
  (OAEI)\footnote{\URL{oaei.ontologymatching.org}}.
This dataset consists of eighty ontologies $G_i$ (for $1 \leq i \leq 80$)
  that are linked to a single base ontology $G_0$.
The identity links between $G_0$ and $G_i$ are annotated with a
  confidence measure between $0.0$ and $1.0$.
A graph $G$ is the result of fully materializing the graph merge
  of $G_i$ (for some $1 \leq i \leq 80$) and $G_0$.
For each of these eighty linked ontologies a reference mapping is available.

In our experiment we ran separate tests for each of the $80$ datasets
  and took the average values for incremental reductions of
  random parts of the identity relations between $G_0$ and the $80$
  different $G_i$'s.
In other words: we deliberately make the identity relation incomplete.
We then calculate the rough set representation using this altered relation.
Subsequently ,we evaluate how many of the removed identity pairs occur in
  the higher approximation.
Our hypothesis is that the percentage of removed \emph{identity pairs}
  in the higher approximation is larger than the percentage of \emph{pairs}
  in the higher approximation.
If the hypothesis is validated, this indicates that
  calculating the rough set representation for a partial identity relation
  would indeed improve suggestions for extending that identity relation.

Since the data may contain noise, using precision degrees $0.0$ and $1.0$
  may be too strict. For this experiment we have set the boundaries
  to $0.05$ and $0.95$ respectively.

Figure \ref{fig:recall_quality} shows the different behaviors of the
  upper and lower approximations, with the upper approximation indeed
  having a dramatically higher recall than the lower approximation.

\begin{figure}
\centering
\includegraphics[width=0.8\linewidth]{./img/recall_quality}
\caption{
  The recall of the lower and higher approximation
    is shown by the green line (circles) and red line (boxes) respectively.
  The quality metric (definition \ref{def:quality})
    is shown by the dashed line.
}
\label{fig:recall_quality}
\end{figure}

In figure \ref{fig:in_higher}
  we see that the randomly removed identity pairs are often
  in the higher approximation, even when large parts of
  the identity relation are removed.

\begin{figure}
\centering
\includegraphics[width=0.8\linewidth]{./img/in_higher}
\caption{
  The percentage of the removed identity pairs that are in the higher approximation
}
\label{fig:in_higher}
\end{figure}

\begin{comment}
Lower recall
1.0,0.37172100183246687,0.24123550044617229,0.23375413992878838,0.19251810221158194,0.1916729296953922,0.1790536768898471,0.17912482945205316,0.1696252362320524,0.17011849853438268
Higher recall
1.0,0.976185052819561,0.9600384749021372,0.9425765362723856,0.9297564861502547,0.9239282325982691,0.9288077551757412,0.900878631549093,0.886987450993775,0.8005503359226204
Quality
0.5190405227795084,0.23261438049447705,0.1904353344556299,0.18505145463363135,0.17658153004935476,0.17387635867899948,0.17168948193128328,0.17017370794837366,0.16844506704419984,0.1677780567227606
Higher cover
0.0006413304103753317,0.0006098495555258293,0.0005825544051162525,0.0005747252831024628,0.0005640718998200667,0.0005626590987592347,0.0005613753926606205,0.0005464403973849621,0.0005372610299329459,0.00046916805031848495
Removed identity pairs in higher
1.0,0.5877680311890837,0.5747530425162004,0.572660333493667,0.5831761670185315,0.5799385908868745,0.5701984042084377,0.5692284399224807,0.558405393333526,0.5051324217787632
\end{comment}



\subsection{Future Experiments}
\label{sec:hypotheses}

Space limitations do not allow us to describe the results of further
  experiments. Instead, we will describe some of the other hypotheses
  that can be evaluated using our approach:
\begin{enumerate}
\item Take an {\small \texttt{owl:sameAs}} relation and
        a {\small \texttt{skos:related}}
        relation defined over the same domain.
      Merge them into a new binary relation $\sim$.
      Establishing the lower and higher approximation of $\sim$,
        the hypothesis is that pairs from {\small \texttt{owl:sameAs}}
        occur more frequently in the lower boundary than pairs from
        {\small \texttt{skos:related}},
\item Take a set of alignment pairs, each associated with a confidence measure
        between $0.0$ and $1.0$.
      Choose an arbitrary cutoff point $0.0 < c < 1.0$.
      The hypothesis is that alignments with a confidence larger than $c$
        occur more frequently in the lower approximation than alignments
        with a confidence smaller than $c$.
\item Take a set of automatically generated alignment pairs with
        associated confidence measures and take the gold standard or
        reference alignment for the same dataset.
      The hypothesis is that pairs that occur in the lower approximation
        of the alignment appear relatively more often in the gold standard
        than pairs that occur in the higher approximation of the alignment.
\item The quality measure $\alpha$ of a reference alignment is generally
        higher than the accuracy measure of an automatically generated
        alignment for the same dataset.
      Or, the quality measure is generally higher for identity relations
        that domain experts consider to be correct.
\end{enumerate}

\noindent Finally, the IIMB datasets are quite small
  (tens of thousands of triples).
We will have to verify whether the current implementation is able to scale
  to bigger datasets.


\section{Extensions}
\label{sec:extensions}

In section \ref{sec:approach} 
  we made some simplifications in order to keep the discussion brief.
We discuss three extensions to our base approach.
The first extension provides further details on how to assess
  identity conditions of typed literals, which may occur in the object
  position of a triple.
The second extension generalizes the notion of a
  predicate, allowing more indiscernibility predicates to be found.
The third extension shows that definitions
  \ref{def:identity_lower_approximation} and
  \ref{def:identity_higher_approximation}
  do not give the most optimal solution in some arcane cases.

\subsection{Identity of typed literals}

In order to deal with predefined datatypes such as integers, dates, etc.,
  RDF introduces the notion of ``typed literals''.
For such typed literals, identity does not suffice in order to ascertain
  that the sets of object terms denote the same resources
  (definition \ref{def:indiscernibility_properties}).
The reason for this is that typed literals have special identity conditions.
Since typed literals are quite common in SW data, we elaborate on this here.
Roughly speaking, for typed literals we assume a lexical-to-value mapping
  \cite{Hayes2004}, which assigns a value to each lexical expression.
For each datatype we assume a datatype-specific identity relation
  partitioning the datatype's value space.
Identity between typed literals is then defined
  as the datatype-specific identity between the values of the literals
  under the lexical-to-value mapping.
This extension was already used to perform the computational
  experiment of section \ref{sec:experiment}.

\begin{comment}
First we assume a datatype map
  \mbox{$D : \mathcal{I} \rightarrow ICEXT(I({\small \texttt{rdfs:Datatype}}))$},
  where $ICEXT$ is the functional map from classes onto their instances.
Second, for each datatype $d$ we assume a lexical-to-value mapping
  $L2V(d)$,\cite{Hayes2004},
  %: V(d) \rightarrow LEX(d)
  which assigns a value to each lexical expression.
Finally, for each datatype $d$ we assume a datatype-specific identity relation
  $\sim_d$ partitioning the datatype's value space $V(d)$.\footnote{
    Relation $\sim_d$ poses some problems to implement correctly,
      see section \ref{sec:implementation} for details.
    }

Suppose that two objects $o_1$ and $o_2$ are both typed literals,
  with $o_1 = \pair{d_1}{x_1}$ and $o_2 = \pair{d_2}{x_2}$
  for datatype names $d_1$ and $d_2$ and value names $x_1$ and $x_2$.
Identity between $o_1$ and $o_2$ is then defined as in
  \ref{def:identity_typed_literals}.

\begin{definition}[Identity for typed literals]
\label{def:identity_typed_literals}
\begin{align}
  o_1 \approx o_1
\,\iff\,
    D(d_1) = D(d_2)
  & \; \land \; &\nonumber\\
    x_1 \in LEX(d_1)
  & \; \land \; &\nonumber\\
    x_2 \in LEX(d_2)
  & \; \land \; &\nonumber\\
    l2v(D(d_1))(x_1) \sim_d l2v(D(d_2))(x_2)\nonumber
\end{align}
\end{definition}

\noindent Notice that the datatype-specific lexical-to-value mapping
  in definition \ref{def:identity_typed_literals} is relevant for
  the identification of identity,
  since two lexical expressions may map onto the same value
  according to one datatype but onto different values
  according to another.
An example of this are the lexical expressions $0.1$ and $0.10000000009$,
  which map to the same value according to datatype
  {\small \texttt{xsd:float}}
  but to different values according to datatype
  {\small \texttt{xsd:decimal}} \cite{Goldberg1991}.

In definition \ref{def:identity_typed_literals}
  the conjuncts which state that the value names belong to
  the respective lexical spaces may seem superfluous at first.
But for ill-typed literals,
  i.e. those whose value names do not belong to the lexical space of
  the specified datatype,
  the interpretation is not determined and they are only known to denote
  some arbitrary non-literal value \cite{Hayes2004}.\footnote{
    From the practice of working with SW data, the authors can testify
    that ill-typed literals do occur and are actually quite common!}
\end{comment}


\subsection{Path-expressions}
\label{sec:path_expressions}

The indiscernibility predicates in
  definition \ref{def:indiscernibility_predicates}
  were assumed to consist of single RDF predicate terms.
This restriction is rather arbitrary.
For instance,
  it may be the case that resources {\small \texttt{dbp:Amsterdam}}
  and {\small \texttt{openei:Amsterdam}} may not share a single property,
  even though the format is located in {\small \texttt{dbp:Netherlands}}
  and the latter is located in {\small \texttt{openei:Netherlands}}
  (but these are not asserted as being the same country).
However, suppose that {\small \texttt{dbp:Netherlands}}
  borders {\small \texttt{dbp:Germany}}
  and {\small \texttt{openei:Netherlands}}
  borders {\small \texttt{openei:Germany}},
  where {\small \texttt{dbp:Germany}} and
  {\small \texttt{openei:Germany}} are asserted to be identical.
If we generalize the notion of a predicate,
  then {\small \texttt{dbp:Amsterdam}} and
  {\small \texttt{openei:Amsterdam}} will share the property
  of being located in a country that borders Germany.
Obviously, Brussels and Brno share this property as well,
  so Brussels, Brno and Amsterdam will be indiscernible with respect to
  this predicate (taken in isolation), but the property is at least
  able to discern Brussels, Brno and Amsterdam from
  Stuttgart, Portland, and The Netherlands.

The notion of a predicate, can be easily generalized
  so that it can be denoted by a sequence of RDF predicate terms.
Such sequences are called \emph{path-expressions} in RDF. 

\begin{comment}
For each sequence of predicate terms $\tuplerange{p_1}{p_n}$
  we assume a functional mapping
  $f_{\tuplerange{p_1}{p_n}} : S_G \rightarrow \powerset{O_G}$,
  called the property sequence mapping
  \mbox{(definition \ref{def:generalized_property_map})}.

\begin{definition}[Property sequence map]
\label{def:generalized_property_map}
\begin{align}
  f_{\tuplerange{p_1}{p_n}}(s)
\,=\,
  \bigsetdef{o \in O_G}{
    \exists_{\range{x_0}{x_n}}(
      x_0 = s \land x_n = o \land\nonumber\\
      \bigwedge_{i=0}^{n-1}\nolimits
          \pair{I(x_i)}{I(x_{i+1})}
        \in
          \bigcup_{p \in \equivset{p_{i+1}}}\nolimits \mathit{Ext}(I(p))
    )
  }\nonumber
\end{align}
\end{definition}

By using property sequence maps,
  the definition for generalized indiscernibility predicates
  is only slightly more complex than its simplified version.

\begin{definition}[Indiscernibility criteria]
\label{def:indiscernibility_criteria}
\begin{align}
  \indp_{\approx}(\set{\range{x_1}{x_n}})
=
  \setdef{
    \tuplerange{p_1}{p_n} \in P_G^n
  }{\nonumber\\
      \exists \range{p_1^1}{p_1^n} \in \equivset{p_1},
    \ldots,
      \exists \range{p_m^1}{p_m^n} \in \equivset{p_n}
    (\nonumber\\
        \equivset{f_{\tuplerange{p_1^1}{p_m^1}} (x_1)}
      =
        \ldots
      =
        \equivset{f_{\tuplerange{p_1^n}{p_m^n}} (x_n)}
    )
  }\nonumber
\end{align}
\end{definition}
\end{comment}

\subsection{Fixpoint definition for approximations}

When using the definitions
  \ref{def:identity_lower_approximation} and
  \ref{def:identity_higher_approximation}
  for determining the rought set represention,
  we do not always get the most optimal solution.
We will illustrate this with the example depicted
  in figure \ref{fig:fixpoint}.

\begin{figure}
\label{fig:fixpoint}
\centering
\includegraphics[width=0.4\columnwidth]{./img/fixpoint_example_cropped}
\caption{
  An illustrative example showing that defining
  $\lowerapprox$ in terms of $\higherapprox$ may not be optimal.
  The identity relation is represented with double-lined edges.
  $s_i$, $p$, and $o_i$ are subject, predicate, and object terms respectively.
  The directed edges represent triples.
}
\end{figure}

The indiscernibility criteria for this example are shown
  in equation \ref{eq:20}.

\begin{align}
\label{eq:20}
  \indp_{\approx}(\set{s_1,s_2,s_3})
&=\\
  \indp_{\approx}(\set{o_1,o_2})
&=
  \setdef{p^n}{\natnum{n}}\nonumber
\end{align}

\noindent Based on these indiscernibility criteria
  the naive lower approximation
  (definition \ref{def:identity_lower_approximation}) is empty.
But notice that there is a slight asymmetry in the way we have
  calculated this result.
We have defined $\lowerapprox$ in terms of $\indp_{\approx}$,
  i.e. in terms of $\approx$.
It seems more correct, however, to base the indiscernibility criteria
  for calculating the lower approximation on $\lowerapprox$ itself
  (and similarly for the higher approximation).
The resulting definition \ref{def:identity_lower_approximation_fixpoint}
  is recursive.
It affects the closure operations over the predicate and object terms
  in \mbox{definition \ref{def:indiscernibility_predicates}}.

\begin{definition}
\label{def:identity_lower_approximation_fixpoint}
\begin{align}
  x \in \lowerapprox
\, \iff \,
    \setdef{y}{x \equiv_{\indp_{\lowerapprox}} y}
  \; \subseteq \;
    \approx\nonumber
\end{align}
\end{definition}

\noindent There are now two correct solutions or fixpoints
  for the present example:

The first solution is the same as for the naive definition;
  $\lowerapprox_1 = \emptyset$,
  where $\indp_{\lowerapprox_1}(X) = \emptyset$
  for $X$ such that $\card{X} > 1$.

The second solution could not be derived in the naive case:
  $\lowerapprox_2 = \set{\pair{s_1}{s_2},\pair{o_1}{o_2}}$,
  with $\indp_{\lowerapprox_2}(\set{s_1,s_2,o_1,o_2}) = \setdef{p^n}{\natnum{n}}$
  and $\indp_{\lowerapprox_1}(X) = \emptyset$
  for all other $X$ such that $\card{X} > 1$.
This is the greatest fixpoint for this example.

Both solutions are correct,
  since both conform to the same strictures imposed by the here presented
  framework.
However, the second solution is better, since a greater fixpoint is
  to be prefered over a smaller fixpoint.
This is in line with our intuitions,
  since this enlarges the number of consistently applied identity pairs,
  as can be glanced from the quality criterion in
  \mbox{definition \ref{def:quality}}.
Reasoning along the same lines,
  a least fixpoint\footnote{
      We have no reason to believe that it should be unique.
    }
  is the best solution for the redefined higher approximation.




\section{Conclusion}
\label{sec:conclusion}

In this paper we presented a new approach for characterizing,
  extending, retracting, and assessing identity relations.
Our approach does this in purely qualitative terms, using schema semantics.

In section \ref{sec:research_goals} we enumerated three research goals.
The first goal is met, since an indiscernibility partition characterizes
  identity subrelations based on the predicates $P$ (closed under identity)
  for which the pairs in that sets are indiscernible.
In this way we can distinguish between different types of identity
  by treating $P$ as a description of a (sub)set of identity pairs.

The second goal is met, since the notion of a rough set allows us to
  distinguish between pairs that must be (lower approximation)
  and those that may be (higher approximation)
  % 'may' = 'not must not'
  in the identity relation.
If we want to add/remove pairs of the identity relation,
  we should not consider pairs of the former but only pairs of
  the latter kind.

The third goal is met, since the measure for rough set accuracy
  is based on the discernibility criteria of an identity set.
The crispness of the set is proportional to the quality of the
  identity relation, based on its semantic consistency.

Our approach provides a new experimental design for evaluating
  hypotheses regarding identity relations that have not been
  evaluated before, in terms of the semantics of the data.
We think that qualitative means of characterizing an identity relation
  is a useful addition to existing quantitative means.
Also, we think that it is more useful and viable to enrich existing
  identity relations in LOD based on the semantics of the datasets
  in which they occur,
  rather than introducing new relationships into SW languages.
Apart from the practical difficulties of teaching practitioners
  and transforming/enriching existing datasets, we suggest that the
  meaning of an identity relation and its subrelations
  is partially defined in its use,
  i.e., in the indiscernibility criteria it embodies.

For our approach it is not necessary to pose additional restrictions
  on a binary relation $\approx$.
The definitions in this paper apply to {\small \texttt{owl:sameAs}} relations
  in the same way in which they apply to other binary relations
  (e.g., {\small \texttt{skos:related}}).

In contemporary ontology alignment and data linking activities,
  non-semantic aspects of resources play a role as well.
For instance, similarity assessment for natural language labels is often
  used in data linking.
We have purposefully not taken such non-semantic aspects into account here,
  also because data linking is not the main focus of our research.
We do see possibilities for future hybrid approaches.

%We are currently in the process of validating the hypotheses
%  enumerated in section \ref{sec:hypotheses}.
%The results of these evaluations are continuously being published on
%  \URL{wouterbeek.com/identity-on-the-web} and
%\URL{github.com/wouterbeek/IOTW}.

%\input{./tex/futurework.tex}

% Bibliography
\bibliographystyle{aaai}
\bibliography{iotw,prasem,web_standards}

\end{document}

