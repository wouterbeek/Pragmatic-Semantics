\subsection{Fixpoint definition for approximations}

When using the definitions
  \ref{def:identity_lower_approximation} and
  \ref{def:identity_higher_approximation}
  for determining the rought set represention,
  we do not always get the most optimal solution.
We will illustrate this with the example depicted
  in figure \ref{fig:fixpoint}.

\begin{figure}
\label{fig:fixpoint}
\centering
\includegraphics[width=0.4\columnwidth]{./img/fixpoint_example_cropped}
\caption{
  An illustrative example showing that defining
  $\lowerapprox$ in terms of $\higherapprox$ may not be optimal.
  The identity relation is represented with double-lined edges.
  $s_i$, $p$, and $o_i$ are subject, predicate, and object terms respectively.
  The directed edges represent triples.
}
\end{figure}

The indiscernibility criteria for this example are shown
  in equation \ref{eq:20}.

\begin{align}
\label{eq:20}
  \indp_{\approx}(\set{s_1,s_2,s_3})
&=\\
  \indp_{\approx}(\set{o_1,o_2})
&=
  \setdef{p^n}{\natnum{n}}\nonumber
\end{align}

\noindent Based on these indiscernibility criteria
  the naive lower approximation
  (definition \ref{def:identity_lower_approximation}) is empty.
But notice that there is a slight asymmetry in the way we have
  calculated this result.
We have defined $\lowerapprox$ in terms of $\indp_{\approx}$,
  i.e. in terms of $\approx$.
It seems more correct, however, to base the indiscernibility criteria
  for calculating the lower approximation on $\lowerapprox$ itself
  (and similarly for the higher approximation).
The resulting definition \ref{def:identity_lower_approximation_fixpoint}
  is recursive.
It affects the closure operations over the predicate and object terms
  in \mbox{definition \ref{def:indiscernibility_predicates}}.

\begin{definition}
\label{def:identity_lower_approximation_fixpoint}
\begin{align}
  x \in \lowerapprox
\, \iff \,
    \setdef{y}{x \equiv_{\indp_{\lowerapprox}} y}
  \; \subseteq \;
    \approx\nonumber
\end{align}
\end{definition}

\noindent There are now two correct solutions or fixpoints
  for the present example:

The first solution is the same as for the naive definition;
  $\lowerapprox_1 = \emptyset$,
  where $\indp_{\lowerapprox_1}(X) = \emptyset$
  for $X$ such that $\card{X} > 1$.

The second solution could not be derived in the naive case:
  $\lowerapprox_2 = \set{\pair{s_1}{s_2},\pair{o_1}{o_2}}$,
  with $\indp_{\lowerapprox_2}(\set{s_1,s_2,o_1,o_2}) = \setdef{p^n}{\natnum{n}}$
  and $\indp_{\lowerapprox_1}(X) = \emptyset$
  for all other $X$ such that $\card{X} > 1$.
This is the greatest fixpoint for this example.

Both solutions are correct,
  since both conform to the same strictures imposed by the here presented
  framework.
However, the second solution is better, since a greater fixpoint is
  to be prefered over a smaller fixpoint.
This is in line with our intuitions,
  since this enlarges the number of consistently applied identity pairs,
  as can be glanced from the quality criterion in
  \mbox{definition \ref{def:quality}}.
Reasoning along the same lines,
  a least fixpoint\footnote{
      We have no reason to believe that it should be unique.
    }
  is the best solution for the redefined higher approximation.
