\documentclass[letterpaper]{article}

% Parts of the style depend on whether a PDF or a DVI output is created.
\usepackage{ifpdf}

\usepackage{aaai}
\usepackage{amsfonts}
\usepackage{amsmath}
\usepackage{amsthm}
\usepackage{courier}
\usepackage[english]{babel}

% Include graphics.
\usepackage{graphicx}
\ifpdf
  % Declare the supported file extensions.
  \DeclareGraphicsExtensions{.jpg,.mps,.pdf,.png}
\fi

\usepackage{helvet}
\usepackage[utf8]{inputenc}
\usepackage{times}
\usepackage{verbatim}
\usepackage{changebar}
\usepackage[normalem]{ulem}
\usepackage{url}
\newcommand{\URL}[1]{{\small \url{#1}}}

% Operator macros.
\newcommand{\absolute}[1]{\lvert#1\rvert}
\newcommand{\bigsetdef}[2]{\big\{#1\,\,\big\vert\,\,#2\big\}}
\newcommand{\card}[1]{\lvert#1\rvert}
\newcommand{\equivpair}[2]{#1 \approx #2}
\newcommand{\equivset}[1]{[#1]_{\approx}}
\newcommand{\higherapprox}[0]{\overline{\approx}}
\newcommand{\interp}[1]{#1^{\mathcal{I}}}
\newcommand{\lowerapprox}[0]{\underline{\approx}}
\newcommand{\natnum}[1]{#1 \in \mathbb{N}}
\newcommand{\pair}[2]{\langle#1,#2\rangle}
\newcommand{\powerset}[1]{\mathcal{P}(#1)}
\newcommand{\range}[2]{#1,\ldots,#2}
\newcommand{\set}[1]{\{#1\}}
\newcommand{\setdef}[2]{\{#1\,\vert\,#2\}}
\newcommand{\setrange}[2]{\{#1,\ldots,#2\}}
\newcommand{\triple}[3]{\langle#1,#2,#3\rangle}
\newcommand{\tuple}[1]{\langle#1\rangle}
\newcommand{\tuplerange}[2]{\langle#1,\ldots,#2\rangle}

% Operator declarations
\DeclareMathOperator{\indpo}{{\mathbb{IND}-\mathbb{PO}}}
\DeclareMathOperator{\indp}{{\mathbb{IND}-\mathbb{P}}}

% Theorem styles.
\newtheorem{assumption}{Assumption}
\newtheorem{axiom}{Axiom}
\newtheorem{convention}{Convention}
\newtheorem{example}{Example}
\newtheorem{definition}{Definition}
\newtheorem{lemma}{Lemma}
\newtheorem{principle}{Principle}
\newtheorem{proposition}{Proposition}
\newtheorem{specification}{Specification}
\newtheorem{statement}{Statement}
\newtheorem{theorem}{Theorem}
%\theoremstyle{definition}

\frenchspacing

\setlength{\pdfpagewidth}{8.5in}
\setlength{\pdfpageheight}{11in}

\pdfinfo{
/Title Rough Set Semantics for Identity on the Web
/Author Wouter Beek, Stefan Schlobach, Frank van Harmelen}
\setcounter{secnumdepth}{1}

\author{
  Wouter Beek \and Stefan Schlobach \and Frank van Harmelen\\
  Vrije Universiteit Amsterdam\\
  De Boelelaan 1081a\\
  1081HV Amsterdam\\
  The Netherlands
}

\title{Rough Set Semantics for Identity on the Web}

\begin{document}

\maketitle
\begin{abstract}
Identity relations are at the foundation of many
  logic-based knowledge representations.
We argue that the traditional notion of equality,
  is unsuited for many realistic knowledge representation settings.
The classical interpretation of equality is too strong
  when the equality statements are re-used outside their original context.
On the Semantic Web, equality statements are used to interlink
  multiple descriptions of the same object,
  using {\small \texttt{owl:sameAs}} assertions.
And indeed, many practical uses of {\small \texttt{owl:sameAs}}
  are known to violate the formal Leibniz-style semantics.

We provide a more flexible semantics to identity by assigning meaning to
  the subrelations of an identity relation
  in terms of the predicates that are used in a knowledge-base.
Using those indiscernability-predicates,
  we define upper and lower approximations of equality in the style of
  rought-set theory, resulting in a quality-measure for
  identity relations.
\end{abstract}

\section{Introduction}
\label{sec:introduction}

Identity relations are a cornerstone of logic-based knowledge representation.
They allow to state and relate properties of an object
  using multiple names for that object, and conversely,
  they allow to infer that different names actually refer to the same object.

In particular, identity relations are
  at the foundation of the Linked Open Data initiative
  and the Semantic Web (SW) in general \cite{BizerCyganiakHeath2007}.
The SW consists of many different sets of assertions
  that are published on the Web by different authors in different locations,
  often using different names for the same object.
Identity relations then allow the interlinking of these multiple descriptions
  of the same thing.
For example, statements about Amsterdam in the DBpedia dataset
  (where Amsterdam is referred to as
  \URL{http://dbpedia.org/resource/amsterdam}, abbreviated as
  \URL{dbp:amsterdam})
  can be combined with statements about Amsterdam in GeoNames
  (where Amsterdam is referred to as
  \URL{http://sws.geonames.rg/2759794}), by asserting identity between
  these two names.

However, the traditional notion of identity
  (expressed by \texttt{owl:sameAs} \cite{MotikPaterschneiderGrau2012})
  is often problematic, e.g. when objects are considered the same in some
  contexts but not in others.
The standing practice in such cases is to use weaker relations of relatedness
  (e.g., \texttt{skos:related} \cite{MilesBechhofer2009}).
Unfortunately, these relations suffer from the opposite problem of having
  almost no formal semantics, thereby limiting reasoners
  in drawing inferences.

According to the traditional semantics of the identity relation,
  identical terms can be replaced for one another in all non-modal contexts
  \emph{salva veritate}.
Practical uses of \texttt{owl:sameAs} are known to violate this
  strict condition
  \cite{HalpinHayes2010,HalpinHayesMccuskerMcguinnessThompson2010}.

\begin{comment}
The SW is not only a formal model,
  but is also a social component that evolves over time,
  i.e. it is a social machine cite{Www2013}.
Being a social and symbolic system at the same time,
  meaning on the SW is denoted by its semantics as well as its pragmatics.
\end{comment}

\subsection{Outline}

In the following section,
  we will analyze in some more detail the problems that
  are caused by the traditional notion of identity, both in general,
  and in the SW setting in particular.
After surveying existing work on these problems
  in section \ref{sec:related_work},
  we will then present our approach to the problem of identity
  in section \ref{sec:approach}.
First at a very high level and then in more formal detail.
We illustrate the results of applying our formalism to
  a realistic heterogeneous SW dataset.
In sections \ref{sec:implementation} and \ref{sec:experimental_design}
  we put our formalism to the test
  in an experimental setting,
  showing that it behaves as required.

\section{Related work}
\label{sec:related_work}

Existing research suggests six different solutions for
  the problem of identity on the SW.

\textbf{[1] Introduce weaker versions of {\small \texttt{owl:sameAs}}}
  \cite{HalpinHayes2010,MccuskerMcguinness2010}.
Candidates for replacement are
  the SKOS concepts
  {\small \texttt{skos:related}} and {\small \texttt{skos:exactMatch}}
  \cite{MilesBechhofer2009}.
The former is not transitive,
  thereby limiting the possibilities for reasoning.
The latter is transitive,
  but can only be used in certain contexts.
It is not defined in what contexts it can be used
  \cite{MilesBechhofer2009}.\footnote{
    For instance, the property {\small \texttt{skos:exactMatch}}
    ``is used to link two concepts, indicating a high degree of confidence
    that the concepts can be used interchangeably across a wide range of
    information retrieval applications.''
  }
\begin{comment}
% SIMILARITY
The problem with using weaker notions such as relatedness,
  is that everything is related to everything in \emph{some} way.}
% Shall we discuss similarity here as well?
% Does similarity differ from relatedness?
\end{comment}

\textbf{[2] Restrict the applicability of identity relations}
  to specific contexts.
In terms of Semantic Web technology, identities are expected to hold
  within a named graph or within a namespace,
  but not necessarily outside of it \cite{HalpinHayes2010}.
\cite{Melo2013} has successfully used the Unique Names Assumption
  within namespaces in order to identify many (arguably) spurious
  identity statements.

\textbf{[3] Introduce additional vocabulary} that does not weaken but extends
  the existing identity relation.
\cite{HalpinHayes2010} mention an explicit distinction that could be made
  between mentioning a term and using a term,
  thereby distinguishing an object and a Web document describing that object.
Other possible extensions of {\small \texttt{owl:sameAs}} might take
  the Fuzzyness and/or uncertainty of identity statements into account.

\textbf{[4] Use domain-specific identity relations}
  \cite{MccuskerMcguinness2010}.
For instance
    ``$x$ and $y$ have the same medical use''
  replaces
    identity in the domain of medicine,
and
    ``$x$ and $y$ are the same molecule''
  replaces
    identity in the domain of chemistry.
The downside to this solution is that domain-specific links are
  only locally valid, thereby limiting knowledge reuse.

\textbf{[5] Change the modeling practice}, possibly in a (semi-)automated way
  by adapting visualization and modeling toolkits to produce notifications
  upon reading SW data, or by posing additional restrictions on the creation
  and alteration of data. For example, adding an RDF link could require
  reciprocal confirmation from the maintainers of the respective datasets.
  \cite{HalpinHayes2010,DingShinavierFininMcguinness2010}
The problem with introducing checks on editing operations,
  is that it violates one of the fundamental underpinnings of the SW;
  namely that on the Web of Data anybody is allowed to say
  anything about anything \cite{AntoniouGrothHarmelenHoekstra2012}.

\textbf{[6] Extract network properties of {\small \texttt{owl:sameAs}}
  datasets} \cite{DingShinavierShangguanMcguinness2010}.
Although this work shows that network analysis can provide insights
  into the ways in which identity is used in the SW,
  these endeavors have not yet been related to the semantics of the
  identity relation.
We believe that utilizing network theoretic aspects in order to
  determine the meaning of identity statements
  would be interesting future research.

What the existing approaches have in common is
  that quite some work has to be done
  (adapting or creating standards, instructing modelers, converting existing
  datasets) in order to resolve only some of the problems of identity.
Our approach provides a way of dealing with the heterogeneous real-world
  usage of identity in the SW that is fully automated and requires
  no changes to standards, modeling practices, or existing datasets.

\section{Approach}
\label{sec:approach}

First we give a short outline of our approach and then
  we provide a more detailed description of the individual steps.

\subsection{Outline of the approach}

We start by assuming that we are given an identity relation $\approx$.
We will then reinterpret this relation as an indiscernibility relation
  relative to different sets of predicates.
Pairs that have the same indiscernibility predicates
  are simi-discernible, i.e.: they discern resources
  based on the same criteria.
Simi-discernibility is an equivalence relation
  which partitions all pairs and thus also the identity relation $\approx$.
The members of the indiscernibility partition
  have a certain overlap with the original identity relation.
The overlap between an indiscernibility subset and the identity relation
  is called an \emph{identity subrelation}.
Each identity subrelation is characterized in terms of predicates
  from the domain vocabulary.
Different forms of identity can therefore be distinguished
  and meaningfully described.
Based on whether there is a complete or a partial overlap
  between the simi-discernible partition members and
  the identity subrelations,
  these partition members belong either to the lower ($\lowerapprox$)
  or to the higher approximation ($\higherapprox$) of $\approx$.
Besides setting a lower and a higher bound to the identity relation,
  we can also calculate the quality of the identity relation
  and the precision of each identity subrelation.

\subsection{Preliminaries}
\label{sec:preliminaries}

%TODO 'resource'->'object'
%TODO explain that 'objects' are not 'object terms'
%     and are not even (always) denoted by object terms.

$G$ denotes an RDF graph. It consists of a set of ground binary
predicates $p(s,o)$, called ``triples'' in Semantic Web jargon, and often
written as $\triple{s}{p}{o}$. These triples form a graph with all
subjects $s$ and objects $o$ as nodes, and 
each assertion $p(s,o)$ corresponding to a directed edge labelled $p$
between $s$ and $o$. 
We assume that graphs are always closed under
  RDFS and OWL-DL entailment.

We identify subsets of RDF terms based on
  their positional occurrence in triples in $G$:
  $S_G$, $P_G$, and $O_G$ denote the subject, predicate and object terms
  in $G$ respectively.

The interpretation $I$ maps RDF terms onto resources,
  and triples onto truth values.
The extension function $Ext$ maps resources onto pairs of resources.
$I(\triple{s}{p}{o})$ is true iff
  $\pair{I(s)}{I(o)} \in Ext(I(p))$ \cite{Hayes2004}.

A note on terminology: Our use of the word `property' does not coincide
  with the notion of an RDF property, but is closer to the notion
  of a FOL property. Our use of the word `relation' will be close to
  both the notion of a FOL relation and the notion of an RDF property.
  We briefly explain the distinctions.

An RDF property is a resource (i.e., a member of the domain)
  that is the interpretation of
  an RDF term that occurs in the predicate position of a triple.\footnote{
    Notice that we do not use the possibility modality in this formulation.
    Every URI \emph{could} occur in the predicate position of a triple,
      but some URIs are known to denote non-property resources.
  }
As the example above shows, the extension of an RDF property is
  a binary \emph{FOL relation},
  i.e., a subset of the Cartesian product of the domain.\footnote{
    In RDF the domain is the set of RDF resources.}

A \emph{FOL property} is a subset of the domain.
The correlate of a FOL property in RDF is a pair that consists of
  a predicate term and an object term (in that order).
The extension of the interpretation of such a pair $\pair{p}{o}$
  is $Ext(I(p))(I(o))$, which is -- indeed -- a set of RDF resources.

In section \ref{sec:path_expressions} we will generalize
  both the concept of a property and the concept of a relation.
Generalized properties are similar to FOL properties
  and generalized binary relations are similar to binary FOL relations,
  but the latter do not correspond to a (single) RDF property.

\begin{comment}
$\equivset{x}$ is the equivalence class for $x$
  under equivalence relation $\approx$.
\end{comment}


\subsection{Reinterpreting identity as indiscernibility}

We start by assuming that we are given an identity relation $\approx$,
  which partitions the subject terms $S_G$ according to
  \mbox{equation} \ref{eq:equivalence_set}.

\begin{equation}
\label{eq:equivalence_set}
  \equivset{x}
=
  \setdef{
    y \in S_G
  }{
    \equivpair{x}{y}
  }
\end{equation}

\noindent Identity can be defined as the smallest equivalence relation,
  i.e. the most fine-grained partition of $S_G$.
For reasoning purposes, the fact that $\approx$ is an equivalence relation
  is important, since this allows for both symmetrical
  and transitive inference.
As we saw in principle \ref{principle:indiscernibility_of_identicals},
  identity implies indiscernibility with respect to all properties.

We can generalize the notion of indiscernibility
  by parameterizing the set of properties with respect to which
  indiscernibility is determined.
According to this generalization,
  resources $x$ and $y$ are indiscernible with respect to
  a set of properties $PO \subseteq P_G \times O_G$
  iff $\forall po \in PO (po(x) \leftrightarrow po(y))$ is the case.

It is important to note that every indiscernibility relation
  is also an equivalence relation, although not necessarily the smallest one.
Moreover, every indiscernibility relation defined over the domain $S_G$
  is also an identity relation,
  just over a different domain \cite{Quine1950}.\footnote{
    For instance, the set of properties
      ``has an income of $x$ euro's a month''
      with $x$ a decimal number, does not uniquely identify people
      (since two people may have the same income),
      but does uniquely identify income groups.
    }

We now reinterpret the identity relation $\approx$,
  as if it were an indiscernibility relation
  whose set of properties $PO$ is implicit.
Based on the extensional specification of the identity relation,
  we can make the set of properties with respect to which
  it is indiscernible explicit.\footnote{
    In Formal Concept Analysis (FCA) literature this set of properties
    is called the intension. The relation between extension and intension
    in FCA is similar to the relation between subject terms and properties,
    but the similarity becomes less apparent when we extend our approach
    to pairs of subject terms (as the `extension') and sets of
    shared predicate terms (as the `intension') later in this section.
    Our thanks go to Cliff Joslyn for pointing this out.
  }
Definition \ref{def:indiscernibility_properties} makes explicit
  the properties relative to which the terms $x_i$ are indiscernibile
  .

%TODO Define for sets or for pairs?
\begin{definition}[Indiscernibility properties]
\label{def:indiscernibility_properties}
\begin{align}
  \indpo_{\approx}(\set{\range{x_1}{x_n}})
=
  \setdef{
    \pair{p}{o} \in P_G \times O_G
  }{\nonumber\\
    \bigwedge_{1 \leq i \leq n}
      \exists p_i \in \equivset{p},
        \exists o_i \in \equivset{o}(
          \triple{x_i}{p_i}{o_i} \in G
        )
% Alternatively, we can place the quantifies outside the conjunction:
%    \exists_{\range{p_1}{p_n} \in \equivset{p}},
%      \exists_{\range{o_1}{o_n} \in \equivset{o}}
%        \bigwedge_{1 \leq i \leq n} \triple{x_i}{p_i}{o_i} \in G
  }\nonumber
\end{align}
\end{definition}

\noindent Using definition \ref{def:indiscernibility_properties},
  we can deduce that the indiscernibility properties for
  the identity pair
  {\small $\pair{\texttt{dbp:Amsterdam}}{\texttt{dbp:Berlin}}$}
  would contain the properties
  ``is a city'' and ``is located in Europe''.

Notice that in definition \ref{def:indiscernibility_properties}
  we close both the predicate terms $p$ and the object terms $o$
  under identity.\footnote{
    It is a common modeling practice in OWL to use equivalence
      instead of identity statements between RDF properties
      (i.e. {\small \texttt{owl:equivalentProperty}})
      and OWL classes
      (i.e. {\small \texttt{owl:equivalentClass}}).
  }
Performing these closures is important in order to identify
  the relevant indiscernibility properties.
Suppose that the RDF resources {\small \texttt{dbp:Berlin}}
  and {\small \texttt{fb:m.0156q}} are identical,
  as are the RDF properties {\small \texttt{schema:containedIn}} and
  {\small \texttt{dbp-owl:locatedInArea}}.
If the database also contains the triples in
  example \ref{ex:triple1} and \ref{ex:triple2},
  we want to be able to identify the pairs in
  {\small $\set{\texttt{schema:containedIn},\texttt{dbp-owl:locatedInArea}}$}
  {\small $\times \set{\texttt{dbp:Berlin},\texttt{fb.0156q}}$}
  as indiscernibility properties.

\small
\begin{example}[Triples]
\begin{align}
\langle
  \texttt{fb:Teufelsberg},\label{ex:triple1}\\
  \texttt{schema:containedIn},\,
  \texttt{dbp:Berlin}
\rangle\nonumber\\
\langle
  \texttt{dbp:Teufelsberg},\label{ex:triple2}\\
  \texttt{dbp-owl:locatedInArea},\,
  \texttt{fb.0156q}
\rangle\nonumber
\end{align}
\end{example}
\normalsize

\noindent In the above, we were interested in the properties
  that resources share with one other
  (e.g., being a city, being located in Europe).
But we are also interested in the predicates that are shared by
  a set of resources.
This amounts to a simple abstraction of
  definition \ref{def:indiscernibility_properties},
  equating the sets of objects (closed under identity)
  and only returning the set of shared RDF predicate terms
  (see definition \ref{def:indiscernibility_predicates}).

\small
\begin{definition}[Indiscernibility predicates]
\label{def:indiscernibility_predicates}
\begin{align}
  \indp_{\approx}(\setrange{x_1}{x_n})
=
  \setdef{
    p \in P_G
  }{
    \exists_{\range{p_1}{p_n} \in \equivset{p}}(\nonumber\\
        \equivset{
          \setdef{
            o \in O_G
          }{
            \triple{x_1}{p_1}{o}
          }
        }
      =
        \ldots
      =
        \equivset{
          \setdef{
            o \in O_G
          }{
            \triple{x_n}{p_n}{o}
          }
        }
    )
  }\nonumber
\end{align}
\end{definition}
\normalsize

\noindent Taking the same example as before, the identity pair
  {\small $\pair{\texttt{dbp:Amsterdam}}{\texttt{dbp:Berlin}}$}
  has the identity closure of ``is a'' and ``is located in''
  as its indiscernibility predicates.



\subsection{Discerning the same}

In the previous section we saw that resources are \mbox{\emph{indiscernible}}
  with respect to $PO$ iff they cannot be told apart
  in a language that only contains the properties denoted by $PO$
  (the so-called indiscernibility properties):

In the same vein,
  and builing upon definition \ref{def:indiscernibility_predicates},
  we say that two pairs of resources are \emph{simi-discernible}
  iff their \mbox{indiscernibility} predicates $P \subseteq P_G$ are the same.

For example, the pair {\small $\pair{\texttt{Amsterdam}}{\texttt{Berlin}}$}
  is simi-discernible to the pair
  {\small $\pair{\texttt{Nanjing}}{\texttt{Dalian}}$},
  because both pairs have the same set of discernibility predicates,
  namely ``is a'' and ``is located in''.

When we look at the pairs that constitute (the extension of)
  an identity relation, all identity assertions look the same.
But when we take the considerations of the previous section into account,
  we see that within a given identity relation
  there are pairs that assert indiscernibility
  based on different domain predicates.
Stating this formally,
  simi-discernibility is an equivalence relation on pairs of resources,
  which induces a partition of the Cartesian product of the domain.
Definition \ref{def:simidiscernibility_relation} makes this concrete
  in terms of the earlier definitions.

\begin{definition}[Simi-discernibility relation]
\label{def:simidiscernibility_relation}
\begin{align}
  \equiv_{\indp_{\approx}}
=
  \setdef{
    \pair{\pair{x_1}{x_2}}{\pair{y_1}{y_2}} \in (S_G^2)^2
  }{\nonumber\\
    \indp_{\approx}(\set{x_1,x_2}) = \indp_{\approx}(\set{y_1,y_2})
  }\nonumber
\end{align}
\end{definition}



\subsection{Partitioning identity}

The members of the partition induced by $\equiv_{\indp_{\approx}}$
  are sets of resource pairs that share the same sharing properties.\footnote{
    We briefly reflect on our wording here.
    \emph{Within} a resource pair, what is shared are properties
      (denoted by a pair of an RDF predicate term and an RDF object term,
      in that order).
    \emph{Between} pairs, what is shared -- if something is shared --
      are predicates, denoted by RDF predicate terms.
  }

Notice that the partitioned pairs contain but are not limited to
  the identity pairs.
Therefore, for sets of pairs closed under simi-discernibility
  we have the following three possibilities:
  \begin{enumerate}
    \item All pairs are identity pairs.
          This characterizes a consistent portion of the identity relation,
          since no simi-discernible pair is left out of this set.
    \item Some pairs are identity pairs.
          This characterizes a portion of the identity relation which is not
          applied consistently with respect to
          the simi-discernibility relation.
    \item No pairs are identity pairs.
          This characterizes a portion of the collection of pairs
          that is consistently kept out of the identity relation.
  \end{enumerate}

\noindent Each member of the simi-discernibility partition that is not
  of the third kind, i.e. every set of pairs that contains some identity pair,
  can be though of as an identity subrelation.
The simi-discernibility partition also partitions the identity relation
  into \emph{identity subrelations}.
Each identity subrelation can be described in terms of
  its discernibility predicates,
  i.e. in meaningful terms drawn from the domain vocabulary.

For instance, in the IIMB dataset that we will use in our experiment
  in sections \ref{sec:implementation} and \ref{sec:experimental_design}
  there are some identical resources that share
  the property {\small \texttt{IIMBTBOX:spoken\_in}},
  while other pairs share the property
  {\small \texttt{IIMBTBOX:form\_of\_government}}.
As a matter of fact, the set of pairs of resources that are spoken in
  the same language (i.e. movies) are disjoint from
  the set of pairs of resources that have the same form of government
  (i.e. countries), indicating identity according to
  different indiscernibility criteria.
In this example, one subset of the identity relation does not discern
  resources that are spoken in the same language,
  whereas another does not discern resources that have
  the same form of government.



\begin{comment}
Fig. 1 shows an example
  of a discernibility partitioning for a given identity relation.

We can thus identify subsets of an identity relation based on
  differences in the sets of properties relative to which
  the resource pairs that they consist of are
  (in)discernible from one another.
Identity of indiscernibility criteria provides
  another equivalence relation
  ($\approx_{\indp}$, def. \ref{def:indiscernibility_partition}),
  that partitions the identity relation $\approx$ into
  identity subrelations that characterize identity based on different
  indiscernibility criteria.

\begin{figure*}
\label{fig:iimb_example}
\centering
\includegraphics[width=\textwidth]{iimb_approximation_example_crop}
\caption{
  An example of a discernibility partition for an identity relation
    consisting of 365 pairs applied to the fourth IIMB linkset.
  Each node is annotated with the set of predicates $P$ for which
    its pairs are $P$-indiscernible.
  The number of identity pairs within each partition set
    is displayed to the right of the predicate set label.
  Partition sets that contain no identity pair are not show.
  The number that occurs to the left of the predicate label in each node
    indicates how may pairs in that node are identity pairs.
  The lower approximation consists of the nodes with a solid border,
    indicating that they contain only identity pairs.
  The higher approximation consists of all displayed nodes.}
\end{figure*}
\end{comment}


\subsection{Quality \& Approximation}
\label{sec:approximation}

Not all identity subrelations have the same quality.
Indeed, when we look at the subdivision into three `categories' above,
  we are able to distinguish between a lower approximation of identity,
  as the union of subrelations from the first category
  (definition \ref{def:identity_lower_approximation}),
  and a higher approximation of identity,
  as the union of subrelations from both the first and the second category
  (definition \ref{def:identity_higher_approximation}).

\begin{definition}[Lower approximation]
\label{def:identity_lower_approximation}
\begin{align}
  x \in \lowerapprox
\, \iff \,
    \setdef{y}{x \equiv_{\indp_{\approx}} y}
  \; \subseteq \;
    \approx\nonumber
\end{align}
\end{definition}

\begin{definition}[Higher approximation]
\label{def:identity_higher_approximation}
\begin{align}
  x \in \higherapprox
\, \iff \,
      \setdef{y}{x \equiv_{\indp_{\approx}} y}
    \, \cap \,
      \approx
  \; \neq \;
    \emptyset\nonumber
\end{align}
\end{definition}

\noindent Based on these approximations we can give
  the rough set representation $\pair{\lowerapprox}{\higherapprox}$
  of identity relation $\approx$.
The quality of a rough set representation is given in
  definition \ref{def:quality}.
The intuition behind this quality measure is that the crispness
  of a set should be proportional to the quality
  of the identity relation on which it is based.
Since a consistently applied identity relation has relatively many
  partition sets that contain either
  no identity pairs (small value for $\lowerapprox$) or
  only identity pairs (large value for $\higherapprox$),
  a more consistent identity relation has a higher accuracy.

\begin{definition}[Quality]
\label{def:quality}
\begin{align}
  \alpha(\approx)
\, = \,
  \dfrac{
    \card{\underline{\sim}}
  }{
    \card{\overline{\sim}}
  }\nonumber
  %\card{\lowerapprox} / \card{\higherapprox}
\end{align}
\end{definition}




\section{Conclusion}
\label{sec:conclusion}

In this paper we presented a new approach for characterizing,
  extending, retracting, and assessing identity relations.
Our approach does this in purely qualitative terms, using schema semantics.

In section \ref{sec:research_goals} we enumerated three research goals.
The first goal is met, since an indiscernibility partition characterizes
  identity subrelations based on the predicates $P$ (closed under identity)
  for which the pairs in that sets are indiscernible.
In this way we can distinguish between different types of identity
  by treating $P$ as a description of a (sub)set of identity pairs.

The second goal is met, since the notion of a rough set allows us to
  distinguish between pairs that must be (lower approximation)
  and those that may be (higher approximation)
  % 'may' = 'not must not'
  in the identity relation.
If we want to add/remove pairs of the identity relation,
  we should not consider pairs of the former but only pairs of
  the latter kind.

The third goal is met, since the measure for rough set accuracy
  is based on the discernibility criteria of an identity set.
The crispness of the set is proportional to the quality of the
  identity relation, based on its semantic consistency.

Our approach provides a new experimental design for evaluating
  hypotheses regarding identity relations that have not been
  evaluated before, in terms of the semantics of the data.
We think that qualitative means of characterizing an identity relation
  is a useful addition to existing quantitative means.
Also, we think that it is more useful and viable to enrich existing
  identity relations in LOD based on the semantics of the datasets
  in which they occur,
  rather than introducing new relationships into SW languages.
Apart from the practical difficulties of teaching practitioners
  and transforming/enriching existing datasets, we suggest that the
  meaning of an identity relation and its subrelations
  is partially defined in its use,
  i.e., in the indiscernibility criteria it embodies.

For our approach it is not necessary to pose additional restrictions
  on a binary relation $\approx$.
The definitions in this paper apply to {\small \texttt{owl:sameAs}} relations
  in the same way in which they apply to other binary relations
  (e.g., {\small \texttt{skos:related}}).

In contemporary ontology alignment and data linking activities,
  non-semantic aspects of resources play a role as well.
For instance, similarity assessment for natural language labels is often
  used in data linking.
We have purposefully not taken such non-semantic aspects into account here,
  also because data linking is not the main focus of our research.
We do see possibilities for future hybrid approaches.

%We are currently in the process of validating the hypotheses
%  enumerated in section \ref{sec:hypotheses}.
%The results of these evaluations are continuously being published on
%  \URL{wouterbeek.com/identity-on-the-web} and
%\URL{github.com/wouterbeek/IOTW}.


% Bibliography
\bibliographystyle{aaai}
\bibliography{iotw,prasem,web_standards}

\end{document}

