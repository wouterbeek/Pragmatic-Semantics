\section{Operationalization of the five star model}
\label{sec:operationalization}

This section gives an operationalized interpretation of
 the five star model of LOD publishing.
The original formulation of the model \cite{Bernerslee2006}
 is displayed in Table \ref{tab:five_star}.
Our goal is to make the model concrete enough in order to be able to
 quantify adherence to it.

\begin{table}
  \label{tab:five_star}
  \caption{Five Star Linked Open Data}
  \centering
  \begin{tabular}{|l|l|}
    \hline
        $\star$
      & Available on the web (whatever format) but with \\
      & an open license, to be Open Data.\\
    \hline
        $\star\star$
      & Available as machine-readable structured data \\
      & (e.g. excel instead of image scan of a table).\\
    \hline
        $\star\star\star$
      & as (2) plus non-proprietary format \\
      & (e.g. CSV instead of excel).\\
    \hline
        $\star\star\star\star$
      & All the above plus, Use open standards from W3C to \\
      & identify things, so that people can point at your stuff.\\
    \hline
        $\star\star\star\star\star$
      & All the above, plus: Link your data to other people's \\
      & data to provide context.\\
    \hline
  \end{tabular}
\end{table}

\begin{principle}[First star]
\label{principle:first_star}
  Available on the web (whatever format) but with an open license,
   to be Open Data.
\end{principle}

\subsubsection*{`stuff'}

The use of the mass noun `stuff` in principle \ref{principle:first_star}
 suggests that what is made available on the Web cannot be readily counted.
Since we want to quantify over data on the Web,
 we replace this word with a count noun
 that allows discrete units to be denoted.
Two replacements seem viable here: datasets and individual parts
 (e.g. files, endpoints) or `resources' that constitute datasets.
We choose resources here,
 because several linked data properties are not defined on datasets,
 but on individual parts.
We define resources as digital entities that have
 been purposefully published with the intent of being (potentially)
 disseminated as some form of partially or completely
 machine-processable data.

\subsubsection*{`Web'}

The Web is interpreted to denote the collection of documents
 (some of them containing data) that are accessable on the Internet.
We restrict ourselves to those parts of the Internet that are
 readily accessable -- i.e. without authentication --
 over TCP/IP by using a standardized Internet protocol
 (mainly FTP and HTTP(S)).

\subsubsection*{`make available'}

There are various aspects to making resources available on the Web.
Availability of a resource for a machine agent means that
 the agent is able to
 (1) locate the resource,
 (2) connect to the host that disseminates the resource, and
 (3) retrieve the resource from the host.

A resource is universally locatable if there exists a resource locator
 that denotes the resource's location
 and this locator can be readily found by a machine agent.
Resource locators have to be supplied to the agent.
This cannot always be done in an automated fashion.
We therefore use a hand-made, curated catalogue of resource locators
 for the automated to use (see section \ref{sec:implementation}).

Findability of the location presupposes the location exists
 and that there is some apparatus residing at that location that is able
 to serve the resource located there.
On the Web existence is a transient notion,
 as locations go in and out of existence all of the time.

The ability of the apparatus at the findable location to serve
the resource consists of
 (3.1) it being responsive to requests from outside,
 (3.2) its ability to interpret requests based on
 the standardized communication protocol that is indicated in
 the universal locator (the so-called `scheme' part of the locator), and
 (3.3) its ability to respond to a correctly formulated request
 in a reply that is itself correctly formulated according to the same
 communication protocol.

As before, we exclude those cases in which it is necessary
 to authenticate and/or use a non-automated way
 of retrieving the resource.
This explicitly excludes content for which a form has to be filled in
 or an account has to be creating by a human user
 in order to gain access.

\subsubsection*{`open license'}

We interpret the requirement of being made available under an open license
 as being made available under a license that is defined by
 the Open Data Commons\footnote{\url{http://opendatacommons.org/}} \cite{miller2008open},
 and that is compliant with either
 the Open Knowledge Foundation's Open Definition\footnote{\url{http://opendefinition.org/od/}}
 or the Open Source Initiative's
 Open Source Definition\footnote{\url{http://opensource.org/license}} \cite{rosen2004open},
 or both.

\begin{principle}[Second star]
  \label{principle:second_star}
  Available as machine-readable structured data
   (e.g. excel instead of image scan of a table).
\end{principle}

\subsubsection*{`structured data'}

For principle \ref{principle:second_star} we interpret
 the structuredness of data as a gradient property,
 and define the amount of structure a collection of data possesses
 as the set of relational operators that are supported by
 the ``common'' implementations of query languages
 over that data.\footnote{The appeal to common implementations
  is needed to rule out exotic cases such as the application
  of set operators on image data.}

This interpretation is difficult to operationalize due to the variety of
 relational operators and the absence of a formal definition of
 ``being common'' or ``being a common implementation''.
Still this interpretation allows us to distinguish between
 an Excel table and an image of the same table,
 since the former allows more queries to be performed than the latter
 (e.g. ordering of rows by the values in the first column,
  summation of the values in the respective columns).

\begin{principle}[Third star]
  \label{principle:third_star}
  as (2) plus non-proprietary format (e.g. CSV instead of excel).
\end{principle}

\subsubsection*{`non-proprietary'}

For principle \ref{principle:third_star} we have not found
 an authoritative source that
 enumerates the open and proprietary formats respectively,
 not even for the common IANA-registered MIME types.
We therefore interpret the proprietary nature of structured data formats
 based on common knowledge that is findable on the Web.

\begin{principle}[Fourth star]
  \label{principle:fourth_star}
  All the above plus, Use open standards from W3C (RDF and SPARQL)
   to identify things, so that people can point at your stuff.
\end{principle}

The use of open standards could  potentially be quantified by the number of
 syntax errors that occur when loading
 a data file in a standards-conformant tool.
However, in practice it is not easy to find tools that are fully
 standards-compliant, and tools differ on the number of erorrs
 they check for an emit.
To lower the dependence on specific tools,
 we do not count the number of syntax errors reported
 but the number of imported triples instead.

We initially wanted to quantify principle \ref{principle:fourth_star}
 by stating that all triples in a dataset ought to be parsable
 and loadable in a triple store.
But we could not find a reliable method for assessing
 the number of triples that is (or should be) present
 in a resource.\footnote{E.g. our investigation showed that
   netiher the CKAN \texttt{size} property
   the number of triples reported in VoID description
   can be used for this, see section \ref{sec:implementation}.}
We therefore decided to interpret successfully passing
 Principle \ref{principle:fourth_star} as
 being able to load some triples (i.e. one or more).

